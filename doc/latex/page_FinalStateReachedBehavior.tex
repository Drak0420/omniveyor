In order to determine the behavior of the trajectory generator after the final state of motion is reached, the flag {\ttfamily \mbox{\hyperlink{classRMLPositionFlags_a85a2739dcace743bf56c81946036d504}{R\+M\+L\+Position\+Flags\+::\+Behavior\+After\+Final\+State\+Of\+Motion\+Is\+Reached}}} can be used. This flag can contain two different values\+:


\begin{DoxyEnumerate}
\item {\ttfamily \mbox{\hyperlink{classRMLPositionFlags_a0cf5bb7ba9fb4d9ab8040b0546170761a4c0ff3d692225145faf8ff420a181a6f}{R\+M\+L\+Position\+Flags\+::\+K\+E\+E\+P\+\_\+\+T\+A\+R\+G\+E\+T\+\_\+\+V\+E\+L\+O\+C\+I\+TY}}} (default)
\item {\ttfamily \mbox{\hyperlink{classRMLPositionFlags_a0cf5bb7ba9fb4d9ab8040b0546170761a1aef826d31747b8d4e1f89c024fb36ae}{R\+M\+L\+Position\+Flags\+::\+R\+E\+C\+O\+M\+P\+U\+T\+E\+\_\+\+T\+R\+A\+J\+E\+C\+T\+O\+RY}}}.
\end{DoxyEnumerate}

The following example has been generated with the Type IV Reflexxes Motion Library and shows the difference between both behaviors.

~\newline
~\newline
 

~\newline
 

~\newline
~\newline
 The left diagram shows the behavior if {\ttfamily \mbox{\hyperlink{classRMLPositionFlags_a85a2739dcace743bf56c81946036d504}{R\+M\+L\+Position\+Flags\+::\+Behavior\+After\+Final\+State\+Of\+Motion\+Is\+Reached}}} is set to {\ttfamily \mbox{\hyperlink{classRMLPositionFlags_a0cf5bb7ba9fb4d9ab8040b0546170761a1aef826d31747b8d4e1f89c024fb36ae}{R\+M\+L\+Position\+Flags\+::\+R\+E\+C\+O\+M\+P\+U\+T\+E\+\_\+\+T\+R\+A\+J\+E\+C\+T\+O\+RY}}}, and the right diagram shows the case of {\ttfamily \mbox{\hyperlink{classRMLPositionFlags_a0cf5bb7ba9fb4d9ab8040b0546170761a1aef826d31747b8d4e1f89c024fb36ae}{R\+M\+L\+Position\+Flags\+::\+R\+E\+C\+O\+M\+P\+U\+T\+E\+\_\+\+T\+R\+A\+J\+E\+C\+T\+O\+RY}}}. In the first case, the desired target velocity is kept until the input values change, and the method \mbox{\hyperlink{classReflexxesAPI_a30d3cdba072553a5d53aa4ce4b4a77d1}{Reflexxes\+A\+P\+I\+::\+R\+M\+L\+Position()}} keeps returning {\ttfamily \mbox{\hyperlink{classReflexxesAPI_a9bf1fa242f1f7005850cd3f517ea8f59a5108326a7898a342063eb558c5614958}{Reflexxes\+A\+P\+I\+::\+R\+M\+L\+\_\+\+F\+I\+N\+A\+L\+\_\+\+S\+T\+A\+T\+E\+\_\+\+R\+E\+A\+C\+H\+ED}}}. In the second case, \mbox{\hyperlink{classReflexxesAPI_a30d3cdba072553a5d53aa4ce4b4a77d1}{Reflexxes\+A\+P\+I\+::\+R\+M\+L\+Position()}} will return {\ttfamily \mbox{\hyperlink{classReflexxesAPI_a9bf1fa242f1f7005850cd3f517ea8f59a5108326a7898a342063eb558c5614958}{Reflexxes\+A\+P\+I\+::\+R\+M\+L\+\_\+\+F\+I\+N\+A\+L\+\_\+\+S\+T\+A\+T\+E\+\_\+\+R\+E\+A\+C\+H\+ED}}} only once (in this example in the control cycle at $ t\ =\ 3075\,ms$), and in the subsequent control cycle, a new trajectory to reach the same state of motion again is computed, which will then be reached in the control cycle at $ t\ =\ 6190\,ms$ (and again at $ t\ =\ 9305\,ms, 12420\,ms, 15535\,ms$, etc.). This behavior repeats until the input values are changed.~\newline
 The position diagram ( $ \vec{P}_i(t)\ =\ \left(\,\!_1p_i(t),\,_2p_i(t)\right) $) also contains the positional extreme values depicted in darker colors (cf. \mbox{\hyperlink{classRMLFlags_a5f2076c59d030993ee522b99873a41cb}{R\+M\+L\+Flags\+::\+Enable\+The\+Calculation\+Of\+The\+Extremum\+Motion\+States}}).~\newline


In this example, a time-\/synchronized trajectory was used. At page \mbox{\hyperlink{page_PSIfPossible}{About the Flag R\+M\+L\+Flags\+::\+P\+H\+A\+S\+E\+\_\+\+S\+Y\+N\+C\+H\+R\+O\+N\+I\+Z\+A\+T\+I\+O\+N\+\_\+\+I\+F\+\_\+\+P\+O\+S\+S\+I\+B\+LE}}, the same example with a phase-\/synchronized motion is shown. ~\newline
~\newline


\begin{DoxyNote}{Note}
{\bfseries{The examples above have been generated with the Type IV Reflexxes Motion Library, which takes into account the values of}}~\newline
~\newline

\begin{DoxyItemize}
\item \mbox{\hyperlink{classRMLInputParameters_a423bf4b1ef337cbf6eee22fe2e2502c1}{R\+M\+L\+Position\+Input\+Parameters\+::\+Current\+Acceleration\+Vector}} (containing the the current acceleration vector $ \vec{A}_i$ and 
\item \mbox{\hyperlink{classRMLInputParameters_a5968ce643868260410f149996c446b66}{R\+M\+L\+Position\+Input\+Parameters\+::\+Max\+Jerk\+Vector}} (containing the maximum jerk vector $ \vec{J}_i^{\,max} $. ~\newline
~\newline
 
\end{DoxyItemize}
\end{DoxyNote}
~\newline
