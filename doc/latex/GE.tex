One more subtle optimization. Why not? Depending on how the GE matrix is constructed, it can be put in a roughly upper-\/triangular form from the start so it looks like this\+: \begin{DoxyVerb}+-------+-------+
| D D D | D D 1 |
| D D D | D 1 M | <- Dense rows
| D D D | 1 M M |
+-------+-------+
| 0 0 1 | M M M |
| 0 0 0 | M M M | <- Sparse deferred rows
| 0 0 0 | M M M |
+-------+-------+
    ^       ^------- Mixing columns
    \--------------- Deferred columns
\end{DoxyVerb}


In the example above, the top 4 rows are dense matrix rows. The last 4 columns are mixing columns and are also dense. The lower left sub-\/matrix is sparse and roughly upper triangular because it is the intersection of sparse rows in the generator matrix. This form is achieved by adding deferred rows starting from last deferred to first, and adding deferred columns starting from last deferred to first.

Gaussian elimination will proceed from left to right on the matrix. So, having an upper triangular form will prevent left-\/most zeros from being eaten up when a row is eliminated by one above it. This reduces row operations. For example, GE on a 64x64 matrix will do on average 100 fewer row operations on this form rather than its transpose (where the sparse part is put in the upper right). 